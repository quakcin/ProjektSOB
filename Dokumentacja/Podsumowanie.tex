\section{Podsumowanie}

Celem projektu było stworzenie systemu transmisji danych o podwyższonej niezawodności, złożonego z ośmiu serwerów połączonych w graf, z centralnym serwerem nadzorującym. Kluczowym elementem projektu było zastosowanie 16-bitowych bloków danych zabezpieczonych kodem Hamminga z dodatkowym bitem parzystości (SECDED), umożliwiającym korekcję pojedynczych błędów i detekcję podwójnych.

\paragraph{}Do realizacji systemu wykorzystano kontenery Docker, co zapewniło izolację środowisk serwerów i ułatwiło zarządzanie nimi. Serwer centralny oraz serwery podrzędne komunikują się poprzez interfejs HTTP, a konfiguracja połączeń między nimi jest dowolna i dynamicznie definiowana. Logika trasowania wiadomości oparta jest na rekursywnym przeszukiwaniu grafu, bez optymalizacji ścieżek.

\paragraph{}Zaimplementowano również mechanizm symulacji błędów, pozwalający użytkownikowi ręcznie wprowadzać błędy w przesyłanych bitach na poziomie każdego serwera. Dzięki temu możliwe było testowanie działania kodu Hamminga w różnych scenariuszach: bez błędów, z pojedynczymi błędami (automatycznie korygowanymi) oraz z podwójnymi błędami (wykrywanymi, ale niekorygowanymi).

\paragraph{}Projekt zawierał też panel do tworzenia i wysyłania komunikatów oraz system logowania operacji, widoczny dla użytkownika w czasie rzeczywistym. Testy wykazały skuteczność systemu w korekcji błędów pojedynczych oraz poprawność obsługi błędów podwójnych, co potwierdziło odporność systemu na zakłócenia transmisji.
