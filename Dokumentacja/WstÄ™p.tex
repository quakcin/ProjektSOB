\section{Wstęp}
\paragraph{Wykonywany temat:} System składający się z 8 serwerów połączonych w graf. Serwer nadzorujący wysyła 16-bitową informację zabezpieczoną kodem korekcyjnym Hamminga, z możliwością symulowania błędów podczas transmisji danych.

\subsection{Założenia projektowe oraz funkcjonalności}
\begin{itemize}
	\item System składa się z 8 serwerów połączonych w graf, umożliwiający komunikację między nimi.
	\item Jeden z serwerów pełni funkcję serwera nadzorującego, który zarządza przesyłaniem danych i kontrolą poprawności transmisji.
	\item Topologia grafu może być dowolna, np. pełne połączenie, pierścień, drzewo, siatka częściowa - struktura będzie definiowana przez użytkownika
	\item Serwer nadzorujący przesyła 16-bitowe bloki danych, które są zabezpieczone kodem korekcyjnym Hamminga 
	\item Kody Hamminga pozwalają na korekcję pojedynczych błędów i detekcję podwójnych błędów, co zwiększa niezawodność systemu.
	\item Każdy serwer odbierający musi być wyposażony w mechanizm dekodowania i sprawdzania syndromu błędu. Po otrzymaniu danych, wykonuje dekodowanie Hamminga, aby sprawdzić poprawność transmisji. W przypadku wykrycia pojedynczego błędu, serwer koryguje go samodzielnie.
	\item Symulacja błędów jest realizowana przez użytkownika dla każdego serwera indywidualnie.
	\item Serwer odbierający raportuje raportuje wystąpienie błędu oraz skuteczność jego korekcji.
	\item Serwer odbierający może być niedostępny podczas próby przesłania żądania, co zostanie ujęte w logu.
\end{itemize}

\subsection{Wstęp teoretyczny}
Celem realizowanego projektu jest stworzenie systemu transmisji danych z ośmioma serwerami połączonymi w grafie, w którym komunikacja odbywa się za pomocą 16-bitowych informacji zabezpieczonych kodem korekcyjnym Hamminga. Aby zapewnić niezawodność transmisji danych, w systemie wykorzystany zostanie kod \textbf{Hamminga z dodatkowym bitem parzystości SECDED} (Single Error Correction, Double Error Detection), który zapewni zarówno wykrywanie, jak i korekcję pojedynczych błędów, a także detekcję podwójnych błędów. 

\paragraph{} Kod Hamminga jest jedną z najczęściej stosowanych metod \textbf{wykrywania i korekcji błędów} w systemach komunikacyjnych i komputerowych. Jego główną zaletą jest zdolność do wykrywania i naprawiania pojedynczych błędów w danych bez konieczności ich retransmisji. Działa to na zasadzie dodania do danych dodatkowych bitów parzystości, które są rozmieszczane w odpowiednich miejscach w danych, umożliwiając w ten sposób identyfikację, który bit uległ zmianie w wyniku błędu transmisji. Dzięki temu, w przypadku wykrycia błędu w jednym z bitów, system jest w stanie go skorygować i przywrócić poprawność danych.

\paragraph{} Jednak klasyczny kod Hamminga, choć skuteczny w przypadku wykrywania i korekcji pojedynczych błędów, nie radzi sobie z wykrywaniem podwójnych błędów. W sytuacjach, gdy w przesyłanych danych wystąpią \textbf{dwa błędy}, klasyczny kod Hamminga \textbf{nie jest w stanie ich wykryć}, co może prowadzić do nieprawidłowej korekcji i w konsekwencji do błędnych danych. Aby rozwiązać ten problem, wprowadzono rozszerzenie kodu Hamminga, zwane SECDED. Jest to modyfikacja klasycznego kodu Hamminga, która dodaje \textbf{dodatkowy bit parzystości}, umożliwiający detekcję dwóch błędów jednocześnie, a jednocześnie pozwala na korekcję pojedynczych błędów. Dzięki temu, kod SECDED zapewnia wyższą niezawodność transmisji, umożliwiając wykrywanie błędów, które mogą wystąpić w wyniku zakłóceń w sieci.

\paragraph{} W omawianym projekcie, serwer nadzorujący będzie odpowiedzialny za wysyłanie 16-bitowych bloków danych, które będą zabezpieczone tym kodem, oraz za symulowanie błędów w transmisji w celu testowania skuteczności tego mechanizmu korekcji. Symulowanie błędów w transmisji pozwala na przeprowadzenie testów w różnych warunkach, sprawdzając, jak system radzi sobie z błędami pojedynczymi oraz podwójnymi, a także w jaki sposób reaguje na ewentualne zakłócenia w sieci.

\subsection{Wybór technologii z stosownym uzasadnieniem}

W projekcie zastosowane zostaną technologie \textbf{Docker} oraz \textbf{PHP} w celu zapewnienia wysokiej niezawodności, elastyczności oraz łatwego zarządzania serwerami i aplikacjami.

\subsubsection{Docker}

Docker to narzędzie do konteneryzacji, które pozwala na uruchamianie aplikacji w odizolowanych środowiskach zwanych kontenerami. Dzięki temu możliwe jest stworzenie spójnego środowiska pracy, niezależnie od infrastruktury serwera. Docker pozwala na łatwe tworzenie, wdrażanie i skalowanie aplikacji w różnych środowiskach.

Zalety użycia Dockera w tym projekcie:
\begin{itemize}
	\item \textbf{Izolacja środowisk} – Każdy serwer w systemie może być uruchomiony w osobnym kontenerze, co zapewnia pełną izolację aplikacji i minimalizuje ryzyko konfliktów między zależnościami.
	\item \textbf{Łatwość skalowania} – Docker umożliwia szybkie tworzenie nowych instancji serwerów (kontenerów), co jest istotne, jeśli w przyszłości projekt będzie musiał obsługiwać większą liczbę serwerów w grafie lub innych rozproszonych systemach.
	\item \textbf{Niezależność od systemu operacyjnego} – Docker zapewnia, że aplikacja będzie działać w ten sam sposób na różnych systemach operacyjnych, co znacząco upraszcza wdrożenia i zarządzanie infrastrukturą.
\end{itemize}


Wybór technologii \textbf{Docker} i \textbf{PHP} jest uzasadniony potrzebą stworzenia skalowalnego, niezawodnego i elastycznego systemu, który będzie łatwy do wdrożenia, zarządzania i testowania. PHP, dzięki swojej elastyczności i wsparciu dla różnych bibliotek, jest odpowiednim wyborem do budowy aplikacji backendowej obsługującej logikę systemu i interakcję pomiędzy serwerami. Te technologie zapewnią stabilność, skalowalność oraz łatwość zarządzania systemem.
